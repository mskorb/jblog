% !TEX TX>S-program = xelatex
% !TEX encoding = UTF-8 Unicode
% This is a simple template for a LaTeX document using the "article" class.
% See "book", "report", "letter" for other types of document.

\documentclass[11pt]{article} % use larger type; default would be 10pt
\usepackage{xunicode}
\usepackage[polish]{polyglossia}
\usepackage{sidecap}
\usepackage[small,bf,it]{caption}
%\usepackage[polish]{babel}

% \usepackage[latin2]{inputenc}%kodowanie literek
% \usepackage[utf8]{inputenc}
%%% Examples of Article customizations
% These packages are optional, depending whether you want the features they provide.
% See the LaTeX Companion or other references for full information.

%%% PAGE DIMENSIONS
\usepackage{geometry} % to change the page dimensions
\geometry{a4paper} % or letterpaper (US) or a5paper or....
% \geometry{margins=2in} % for example, change the margins to 2 inches all round
% \geometry{landscape} % set up the page for landscape
%   read geometry.pdf for detailed page layout information

 
\usepackage{graphicx} % support the \includegraphics command and options
\usepackage{wrapfig}

\usepackage{floatflt}

% \usepackage[parfill]{parskip} % Activate to begin paragraphs with an empty line rather than an indent

%%% PACKAGES
\usepackage{booktabs} % for much better looking tables
\usepackage{array} % for better arrays (eg matrices) in maths
\usepackage{paralist} % very flexible & customisable lists (eg. enumerate/itemize, etc.)
\usepackage{verbatim} % adds environment for commenting out blocks of text & for better verbatim
\usepackage{subfig} % make it possible to include more than one captioned figure/table in a single float
% These packages are all incorporated in the memoir class to one degree or another...

%%% HEADERS & FOOTERS
\usepackage{fancyhdr} % This should be set AFTER setting up the page geometry
\pagestyle{fancy} % options: empty , plain , fancy
\renewcommand{\headrulewidth}{0pt} % customise the layout...
\lhead{}\chead{}\rhead{}
\lfoot{}\cfoot{\thepage}\rfoot{}

%%% SECTION TITLE APPEARANCE
\usepackage{sectsty}
\allsectionsfont{\sffamily\mdseries\upshape} % (See the fntguide.pdf for font help)
% (This matches ConTeXt defaults)

%%% ToC (table of contents) APPEARANCE
\usepackage[nottoc,notlof,notlot]{tocbibind} % Put the bibliography in the ToC
\usepackage[titles,subfigure]{tocloft} % Alter the style of the Table of Contents
\renewcommand{\cftsecfont}{\rmfamily\mdseries\upshape}
\renewcommand{\cftsecpagefont}{\rmfamily\mdseries\upshape} % No bold!

%%% END Article customizations

%%% The "real" document content comes below...
\title{Tytul}
\author{Autor}
%\date{} % Activate to display a given date or no date (if empty),
         % otherwise the current date is printed 

\begin{document}
\maketitle

\section{Rozdzial}
\begin{wrapfigure}{r}{2in}
  \begin{center}
    \includegraphics[width=1.8in]{1.jpg}
  \end{center}
  \caption{Przykladowy obrazek}
\end{wrapfigure}
Tekst Tekst Tekst Tekst Tekst Tekst Tekst Tekst Tekst Tekst Tekst Tekst Tekst Tekst Tekst Tekst Tekst Tekst Tekst Tekst Tekst Tekst Tekst Tekst Tekst Tekst Tekst Tekst Tekst Tekst Tekst Tekst Tekst Tekst Tekst Tekst Tekst Tekst Tekst Tekst Tekst Tekst Tekst Tekst Tekst Tekst Tekst Tekst Tekst Tekst Tekst Tekst Tekst Tekst Tekst Tekst Tekst Tekst Tekst Tekst Tekst Tekst Tekst Tekst Tekst Tekst Tekst Tekst Tekst Tekst Tekst Tekst Tekst Tekst Tekst Tekst Tekst Tekst Tekst Tekst Tekst Tekst Tekst Tekst Tekst Tekst Tekst Tekst Tekst Tekst 

\subsection{Podrozdzial}

Tekst Tekst Tekst Tekst Tekst Tekst Tekst Tekst Tekst Tekst Tekst Tekst Tekst Tekst Tekst Tekst Tekst Tekst Tekst Tekst Tekst Tekst Tekst Tekst Tekst Tekst Tekst Tekst Tekst Tekst Tekst Tekst Tekst Tekst Tekst Tekst Tekst Tekst Tekst Tekst Tekst Tekst Tekst Tekst Tekst Tekst Tekst Tekst Tekst Tekst Tekst Tekst Tekst Tekst Tekst Tekst Tekst Tekst Tekst Tekst Tekst Tekst Tekst Tekst Tekst Tekst Tekst Tekst Tekst Tekst Tekst Tekst Tekst Tekst Tekst Tekst Tekst Tekst Tekst Tekst Tekst Tekst Tekst Tekst Tekst Tekst Tekst Tekst Tekst Tekst 


\begin{enumerate}
\item Lista numerowana:
\item cialo listy numerowanej
    \begin{itemize}
      \item $A_{0}$  - Wypunktowana lista wewnatrz listy numerowanej
       \item $\rho$  - Przykładowy tekst
       \item $s$ - Przykładowy tekst
       \item $\omega$ - Przykładowy tekst
      \item $g_{0}$ - Przykładowy tekst
    \end{itemize}
\end{enumerate}



\subsection{Podrozdzial drugi}
Tekst Tekst Tekst Tekst Tekst Tekst Tekst Tekst Tekst Tekst Tekst Tekst Tekst Tekst Tekst Tekst Tekst Tekst Tekst Tekst Tekst Tekst Tekst Tekst Tekst Tekst Tekst Tekst Tekst Tekst Tekst Tekst Tekst Tekst Tekst Tekst Tekst Tekst Tekst Tekst Tekst Tekst Tekst Tekst Tekst Tekst Tekst Tekst Tekst Tekst Tekst Tekst Tekst Tekst Tekst Tekst Tekst Tekst Tekst Tekst Tekst Tekst Tekst Tekst Tekst Tekst Tekst Tekst Tekst Tekst Tekst Tekst Tekst Tekst Tekst Tekst Tekst Tekst Tekst Tekst Tekst Tekst Tekst Tekst Tekst Tekst Tekst Tekst Tekst Tekst 


\section{Rozdzial}
    \begin{itemize}
      \item http://wikipedia.pl ( Lista numerowana)
    \end{itemize}
\end{document}
